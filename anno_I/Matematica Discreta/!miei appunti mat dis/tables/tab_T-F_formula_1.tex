% !TEX root=../Matematica discreta.tex

\begin{table}[h]
\centering
\caption{Tabella di verità dell'\ref{eq:formula_1}}\label{tab:T-F_formula_1}
\begin{tabular}[t]{ c c c c c c c c c } \toprule
    \(p\) & \(q\) & \(r\)
        & \(p \longrightarrow r\) & \(p \land (q \longrightarrow r)\)
            & \(\neg q\) & \(\neg r\) & \(\neg q \lor \neg r\)
                & \(\big(p \land (q \longrightarrow r)\big) \longleftrightarrow (\neg q \lor \neg r)\) \\ \midrule
    \true[b]  &\true[b]  &\true[b]  &\true[b]  &\true[b]  &\false[b] &\false[b] &\false[b] &\false[b] \\
    \true[b]  &\true[b]  &\false[b] &\false[b] &\false[b] &\false[b] &\true[b]  &\true[b]  &\false[b] \\
    \true[b]  &\false[b] &\true[b]  &\true[b]  &\true[b]  &\true[b]  &\false[b] &\true[b]  &\true[b]  \\
    \true[b]  &\false[b] &\false[b] &\true[b]  &\true[b]  &\true[b]  &\true[b]  &\true[b]  &\true[b]  \\
    \false[b] &\true[b]  &\true[b]  &\true[b]  &\false[b] &\false[b] &\false[b] &\false[b] &\true[b]  \\
    \false[b] &\true[b]  &\false[b] &\false[b] &\false[b] &\false[b] &\true[b]  &\true[b]  &\false[b] \\
    \false[b] &\false[b] &\true[b]  &\true[b]  &\false[b] &\true[b]  &\false[b] &\true[b]  &\false[b] \\
    \false[b] &\false[b] &\false[b] &\true[b]  &\false[b] &\true[b]  &\true[b]  &\true[b]  &\false[b] \\ \bottomrule
\end{tabular}
\end{table}