% !TEX root = ../Matematica discreta.tex

\begin{figure}[h]
\centering
    \begin{tikzpicture}
        % draw the axis
        \draw[-latex] (-1, 0) -- (3, 0);   % x axis
        \draw[-latex] (0, -1) -- (0, 3);   % y axis

        % where 1 and origin is on axis
        \node (1-x) [below]    at (1, 0) {\(1\)};
        \node (1-y) [left]     at (0, 1) {\(1\)};
        \node (0) [below left] at (0, 0) {\(0\)};

        % x0 and y0 coords
        \node (x0) [below] at (2, 0) {\(x_0\)};
        \node (y0) [left]  at (0, 2) {\(y_0\)};

        % the meeting point
        \coordinate (x0-y0) at (2, 2);
        \node at (x0-y0) [above right] {\((x_0, y_0)\)};

        % draw lines and dot
        \draw[dashed, thin] (x0) -- (x0-y0) -- (y0);
        \filldraw[black] (x0-y0) circle (1pt);
    \end{tikzpicture}
\caption{Generico punto di coordinate \((x_0, y_0)\) all'interno del piano cartesiano \(\R \times \R\)}\label{fig:punto_piano_cartesiano}
\end{figure}