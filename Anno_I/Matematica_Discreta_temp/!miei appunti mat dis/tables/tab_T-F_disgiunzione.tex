% !TEX root = ../Matematica discreta.tex

\begin{table}[h]
\centering
\caption{Tabella della verità della \textsc{disgiunzione} (\(\lor\))}\label{tab:T-F_disgiunzione}
    \begin{tabular}[t]{ c c c }                               \toprule
        \(a\) & \(b\) & \(a \lor b\)                       \\ \midrule
          \true[b]{}   &   \true[b]{}    &   \true[b]{}    \\
          \true[b]{}   &   \false[b]{}   &   \true[b]{}    \\
          \false[b]{}  &   \true[b]{}    &   \true[b]{}    \\
          \false[b]{}  &   \false[b]{}   &   \false[b]{}   \\ \bottomrule
    \end{tabular}
\end{table}